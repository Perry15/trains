\begin{acronym}

\acro{ICT}[ICT]{Information and Communications Technology}%{\small L'insieme dei metodi e delle tecniche utilizzate nella trasmissione, ricezione ed elaborazione di dati e informazioni.\par}

\acro{IT}[IT]{Information Technology}%{\small Informatica.\par}

\acro{BSS}[BSS]{Business Support System}%{\small Sistemi che un operatore telefonico utilizza per gestire le operazioni commerciali nei confronti dei suoi clienti.\par}

\acro{EAI}[EAI]{Enterprise Application Integration}%{\small Processo d'integrazione tra diversi tipi di sistemi informatici attraverso l'utilizzo di software e soluzioni architetturali.\par}

\acro{SOA}[SOA]{Service-Oriented Architecture}%{\small Un paradigma per l'organizzazione e l'utilizzo delle risorse distribuite che possono essere sotto il controllo di domini di proprietà differenti. Fornisce un mezzo uniforme per offrire, scoprire, interagire ed usare le capacità di produrre gli effetti voluti consistentemente con presupposti e aspettative misurabili.\par}

\acro{RD}[R\&D]{Research and Development}%{\small La locuzione viene usata per indicare quella parte di un'impresa (persone, mezzi e risorse finanziarie), che viene dedicata allo studio di innovazione tecnologica da utilizzare per migliorare i propri prodotti, crearne di nuovi, o migliorare i processi di produzione.\par}

\acro{IoT}[IoT]{Internet of Things}%{\small L'Internet delle cose è una possibile evoluzione dell'uso della Rete: gli oggetti (le "cose") si rendono riconoscibili e acquisiscono intelligenza grazie al fatto di poter comunicare dati su se stessi e accedere ad informazioni aggregate da parte di altri.\par}

\acro{DPO}[DPO]{Data Protection Officier}%{\small Specialista che conosce in modo specifico sia i sistemi informatici di storage che le normative in materia, con potere di intraprendere azioni per la sicurezza dei dati custoditi.\par}

\acro{GDPR}[GDPR]{General Data Protection Regulation}%{\small Regolamento dell'Unione europea in materia di trattamento dei dati personali e di privacy. Con questo regolamento, la Commissione europea si propone come obiettivi quello di rafforzare la protezione dei dati personali di cittadini dell'Unione europea e dei residenti nell'Unione europea, sia all'interno che all'esterno dei confini dell'Unione europea (UE), restituendo ai cittadini il controllo dei propri dati personali, semplificando il contesto normativo che riguarda gli affari internazionali, unificando e rendendo omogenea la normativa privacy dentro l'UE.\par}

\acro{BIG-ASC}[BIG-ASC]{Big Data and Advanced Analytics for Secure Mobile Commerce}%{\small Progetto di Sync Lab che ha come obbiettivo la creazione di una piattaforma Big Data che sappia rispondere a requisiti stringenti delle piattaforme di Mobile Commerce, come scalabilità, autonomia e performance.\par}

\acro{SMS}[SMS]{Skill Management System}%{\small Portale sviluppato da Sync Lab che consente di amministrare i profili delle persone interessate a un colloquio.\par}

\acro{SE}[SE]{Standard Edition}%{\small Piattaforma software ampiamente utilizzata nella programmazione in linguaggio Java per costruire e distribuire applicazioni portatili a uso generale.\par}

\acro{OOP}[OOP]{Object Oriented Programming}%{\small Programmazione orientata agli oggetti.\par}

\acro{PCD}[PCD]{Programmazione Concorrente e Distribuita}%{\small Programmazione concorrente e distribuita.\par}

\acro{MVC}[MVC]{Model View Controller}%{\small Pattern architetturale molto diffuso nello sviluppo di sistemi software, in particolare nell'ambito della programmazione orientata agli oggetti, in grado di separare la logica di presentazione dei dati dalla logica di business.\par}

\acro{REST}[REST]{Representational State Transfer}%{\small L'architettura REST si basa su HTTP; il funzionamento prevede una struttura degli URL ben definita (atta a identificare univocamente una risorsa o un insieme di risorse) e l'utilizzo dei verbi HTTP specifici per il recupero di informazioni (GET), per la modifica (POST, PUT, PATCH, DELETE) e per altri scopi (OPTIONS, ecc.).\par}

\acro{API}[API]{Application Programming Interface}%{\small Insieme di funzioni (in genere raggruppate per strumenti specifici) usate per portare a termine un compito.\par}

\acro{AMQP}[AMQP]{Advanced Message Queueing Protocol}%{\small Standard aperto che definisce un protocollo a livello applicativo per middleware che gestiscono messaggi. AMQP è definito in modo tale da garantire funzionalità di messaggistica, accodamento, routing (con paradigmi punto-punto e pubblicazione-sottoscrizione), affidabilità e sicurezza.\par}

\acro{DBMS}[DBMS]{Database Management System}%{\small Sistema software progettato per consentire la creazione, la manipolazione e l'interrogazione efficiente di database.\par}

\acro{NoSQL}[NoSQL]{Not Only SQL}%{\small Il nome etichetta il crescente numero di database non relazionali e distribuiti che spesso non forniscono le classiche caratteristiche ACID: atomicità, coerenza, isolamento, durabilità. Il motivo per il quale tali caratteristiche non sono fornite è il cosiddetto teorema CAP.\par}

\acro{JSP}[JSP]{JavaServer Pages}%{\small Tecnologia di programmazione Web in Java per lo sviluppo della logica di presentazione (tipicamente secondo il pattern MVC) di applicazioni Web, fornendo contenuti dinamici in formato HTML o XML. Si basa su un insieme di speciali tag, all'interno di una pagina HTML, con cui possono essere invocate funzioni predefinite sotto forma di codice Java (JSTL) e/o funzioni JavaScript.\par}

\acro{ACID}[ACID]{Atomicity, Consistency, Isolation, Durability}%{\small Proprietà logiche che devono avere le transazioni.\par}

\acro{DRY}[DRY]{Don't Repeat Yourself}%{\small Principio di progettazione e sviluppo secondo cui andrebbe evitata ogni forma di ripetizione e ridondanza logica nell'implementazione di un sistema software.\par}

\acro{ESB}[ESB]{Enterprise Service Bus}%{\small Infrastruttura software che fornisce servizi di supporto a service-oriented architecture complesse. Si basa su sistemi disparati, interconnessi con tecnologie eterogenee, e fornisce in maniera consistente servizi di coordinamento, sicurezza, messaggistica, instradamento intelligente e trasformazioni, agendo come una dorsale attraverso la quale viaggiano servizi software e componenti applicativi.\par}

\acro{CFU}[CFU]{Crediti Formativi Universitari}%{\small Modalità utilizzata nelle università italiane per misurare il carico di lavoro richiesto allo studente per il conseguimento di un diploma di laurea.\par}

\end{acronym}