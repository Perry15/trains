\chapter{Introduzione\label{sec:introduzione}}
\lettrine[findent=1.5em]{L}a relazione descrive il progetto di Mobile Programming svolto da Matteo Marchiori e Giovanni Peron e consiste in un'applicazione mobile con elementi di gamification.

\section{Obiettivi del progetto\label{sec:obiettivi}}
Gli obiettivi del progetto sono:
\begin{itemize}
    \item sviluppare un'applicazione mobile cross-platform;
    \item studiare e implementare alcune tecniche di gamification.
\end{itemize}

\section{Descrizione generale\label{sec:descrizione}}
IsTrainFree è un'applicazione mobile cross-platform utile agli utenti che usano i treni regionali per viaggiare da un posto all'altro. Dopo aver scelto il treno che vuole prendere, l'utente diventa un giudice e dà un voto alla carrozza del treno in cui si trova. Ogni utente può vedere la media dei voti dati al treno fino a quel momento, in modo da capire se conviene prendere quel treno o un altro, oppure cambiare mezzo di trasporto.

Gli elementi di gamification introdotti servono per coinvolgere l'utente e renderlo abituato all'uso dell'applicazione, di modo da farlo sentire soddisfatto e in modo da far crescere il numero di utenti che usano l'applicazione soprattutto in fase iniziale.
Un altro obiettivo importante è quello di aiutare l'utente a capire se ci sono treni utili in fasce orarie inaspettate, in modo da ottimizzare eventuali viaggi e da far usare di più i treni in alternativa a mezzi di trasporto più inquinanti.

\section{Utenti target}
Il pubblico di utenti che abbiamo considerato è generico e vario, perché chiunque può prendere un treno.
L'applicazione vuole fornire un servizio ai viaggiatori che abitualmente prendono già il treno e coinvolgere e invogliare gli utenti meno abituali ad usare un mezzo più ecosostenibile rispetto ad altri.